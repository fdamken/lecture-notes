%%
% Copyright 2016 Fabian Damken
% 
% Licensed under the Apache License, Version 2.0 (the "License");
% you may not use this file except in compliance with the License.
% You may obtain a copy of the License at
% 
%     http://www.apache.org/licenses/LICENSE-2.0
% 
% Unless required by applicable law or agreed to in writing, software
% distributed under the License is distributed on an "AS IS" BASIS,
% WITHOUT WARRANTIES OR CONDITIONS OF ANY KIND, either express or implied.
% See the License for the specific language governing permissions and
% limitations under the License.
%%



\documentclass[12pt]{scrreprt}

\usepackage{amsfonts}
\usepackage[german]{babel}
\usepackage[T1]{fontenc}
\usepackage[utf8]{inputenc}
\usepackage{mathtools}

\title{Mathe 1}
\subtitle{Mitschrift}
\author{Fabian Damken}
\date{\today}

\begin{document}
    \maketitle
    \tableofcontents

    \chapter{Grundbegriffe}
        \label{c:grundbegriffe}

        \section{Aussagen}
            \label{s:grundbegriffe_aussagen}

            Beispiele:
            \begin{itemize}
                \item $ A _ 1 $: $ 3 $ ist eine gerade Zahl.
                \item $ A _ 2 $: Jede natürliche Zahl ist gerade.
                \item $ A _ 3 $: $ 3 $ ist prim.
            \end{itemize}


            \subsection{Aussageformen}
                \label{ss:grundbegriffe_aussagen_formen}

                Aussagen mit Variablen.

                Beispiele:
                \begin{itemize}
                    \item $ E _ 1 $: $ x + 10 = 5 $
                    \item $ E _ 2 $: $ x ^ { 2 } >= 0 $ 
                    \item $ E _ 3 $: $ n $ ist gerade.
                    \item $ E _ 4 $: $ x ^ { 2 } + y ^ { 2 } = 1 $
                \end{itemize}


            \subsection{Quantoren}
                \label{ss:grundbegriffe_aussagen_quantoren}

                \begin{itemize}
                    \item $ \forall x \in M : E(x) $ - Für alle $ x $ in $ M $ gilt $ E(x) $ wobei $ E $ eine Aussageform darstellt.
                    \item $ \exists x \in M : E(x) $ - Es existiert mindestens ein $ x $ in $ M $ für das gilt $ E(x) $ wobei $ E $ eine Aussageform darstellt.
                \end{itemize}

                Beispiele:
                \begin{itemize}
                    \item $ \forall x \in \mathbb{R} : x ^ 2 >= 0 $ - (w)
                    \item $ \forall n \in \mathbb{N} : E _ 3 (n) $ - (f)
                    \item $ \exists n \in \mathbb{N} : E _ 3 (n) $ - (w)
                \end{itemize}

            \subsection{Aussagenlogische Verknüpfungen}
                \label{ss:grundbegriffe_aussagen_verknuepfungen}

                \begin{itemize}
                    \item $ A \land B $ - Konjunktion (und)
                    \item $ A \lor B $ - Disjunktion (oder)
                    \item $ A \implies B $ - Implikation (aus $ A $ folgt $ B $)
                    \item $ \lnot A $ - Negation (nicht)
                    \item $ A \iff B $ - Äquivalenz (Gleichheit)
                \end{itemize}

                \begin{tabular}{ c | c | c | c | c | c | c | c}
$ A $   & $ B $ & $ \lnot A $   & $ \lnot B $   & $ A \land B $ & $ A \lor B $  & $ A \implies B $ ($ (\lnot A) \lor B $)   & $ A \iff B $  \\
\hline
w       & w     & f             & f             & w             & w             & w                                         & w             \\
w       & f     & f             & w             & f             & w             & f                                         & f             \\
f       & w     & w             & f             & f             & w             & w                                         & f             \\
f       & f     & w             & w             & f             & f             & w                                         & f             \\
                \end{tabular}

                \begin{itemize}
                    \item[Äquivalenz] $ A \iff B \equiv (A \implies B) \land (B \implies A) $
                    \item[Kontraposition] $ A \implies B \iff (\lnot B \implies \lnot A) $
                \end{itemize}


                \subsubsection{de Morgan'schen Regeln}
                    \label{sss:grundbegriffe_aussagen_verknuepfungen_deMorgan}

                    \begin{itemize}
                        \item $ \lnot (A \lor B) \iff \lnot A \land \lnot B $
                        \item $ \lnot (A \land B) \iff \lnot A \lor \lnot B $
                    \end{itemize}


                \subsubsection{Distributivgesetz}
                    \label{sss:grundbegriffe_aussagen_verknuepfungen_distributivgesetz}

                    \begin{itemize}
                        \item $ (A \lor B) \land C \iff (A \land C) \lor (B \land C) $
                        \item $ (A \land B) \lor C \iff (A \lor C) \land (B \lor C) $
                    \end{itemize}



        \section{Mengen}
            \label{s:grundbegriffe_mengen}

            Beispiele:
            \begin{itemize}
                \item $ \mathbb{N} = \{ 0; 1; ...; n; ... \} $
                \item $ \mathbb{N*} = \{ 1; 2; ...; n; ... \} = \{ n \in \mathbb{N} : n \neq 0 \} $
                \item $ \{ x \in M : E(x) \} $ wobei $ E $ eine Aussagenform darstellt.
                \item $ \{ n \in \mathbb{N} : prim(x) \land n <= 6 \} = \{ 2; 3; 5 \} $
            \end{itemize}

            \subsection{Formalia}
                \label{ss:grundbegriffe_mengen_formalia}

                \begin{itemize}
                    \item $ A \subseteq B \equiv \forall x \in A : x \in B $
                    \item $ A = B \equiv (A \subseteq B) \land (B \subseteq A) \equiv \forall x \in M : (x \in A \implies x \in B ) \land (x \in B \implies x \in A) $
                    \item $ \emptyset \equiv \{ x \in A : x \neq x \} $ ($ x \neq x \equiv \lnot x = x $)
                \end{itemize}


            \subsection{Operationen}
                \label{ss:grundbegriffe_mengen_operationen}

                $ M, N \in G $

                \begin{itemize}
                    \item $ M \cap N \equiv \{ x \in M : x \in N \} \equiv \{ x \in G : x \in M \land x \in N \} $
                    \item $ M \cup N \equiv \{ x \in G : x \in M \lor x \in N \} $
                    \item $ M \setminus N \equiv \{ x \in M : x \not\in N \} \equiv \{ x \in M : \lnot x \in N \} $
                    \item $ M ^ \mathsf{c} \equiv \{ x \in G : x \not\in M \} \equiv \{ x \in G : \lnot x \in M \} $
                    \item $ M \times N \equiv \{ (x, y) : x \in M , y \in N \} $ - Kartesisches Produkt
                    \item $ A _ 1 \times ... \times A _ n \equiv \{ (x _ 1, ..., x _ n) : x _ y \in A _ 1, ..., x _ n \in A _ n \} $
                    \item $ P(M) = \{ x : x \in M \} $
                    \begin{itemize}
                        \item $ \emptyset \subseteq P(\emptyset) \subseteq P(P(\emptyset)) \subseteq ... $
                        \item $ V _ w \subseteq P ^ n (\emptyset) $ ($ n \in \mathbb{N} $)
                        \item $ P(V _ w) = V _ (w + 1) $
                    \end{itemize}
                \end{itemize}
\end{document}
