%%
% Copyright 2016 Fabian Damken
% 
% Licensed under the Apache License, Version 2.0 (the "License");
% you may not use this file except in compliance with the License.
% You may obtain a copy of the License at
% 
%     http://www.apache.org/licenses/LICENSE-2.0
% 
% Unless required by applicable law or agreed to in writing, software
% distributed under the License is distributed on an "AS IS" BASIS,
% WITHOUT WARRANTIES OR CONDITIONS OF ANY KIND, either express or implied.
% See the License for the specific language governing permissions and
% limitations under the License.
%%



\documentclass[12pt]{scrreprt}

\usepackage{amsfonts}
\usepackage[german]{babel}
\usepackage{eqnarray}
\usepackage[T1]{fontenc}
\usepackage[utf8]{inputenc}
\usepackage{mathtools}
\usepackage{stmaryrd}

\title{Mathe 1}
\subtitle{Mitschrift}
\author{Fabian Damken}
\date{\today}

\setcounter{secnumdepth}{3}

\newcommand{\qed}{\begin{flushright}q.e.d.\end{flushright}}

\begin{document}
    \maketitle
    \tableofcontents

    \chapter{Grundbegriffe}
        \label{c:grundbegriffe}

        \section{Aussagen}
            \label{s:grundbegriffe_aussagen}

            Beispiele:
            \begin{itemize}
                \item $ A _ 1 $: $ 3 $ ist eine gerade Zahl.
                \item $ A _ 2 $: Jede natürliche Zahl ist gerade.
                \item $ A _ 3 $: $ 3 $ ist prim.
            \end{itemize}


            \subsection{Aussageformen}
                \label{ss:grundbegriffe_aussagen_formen}

                Aussagen mit Variablen.

                Beispiele:
                \begin{itemize}
                    \item $ E _ 1 $: $ x + 10 = 5 $
                    \item $ E _ 2 $: $ x ^ { 2 } >= 0 $ 
                    \item $ E _ 3 $: $ n $ ist gerade.
                    \item $ E _ 4 $: $ x ^ { 2 } + y ^ { 2 } = 1 $
                \end{itemize}


            \subsection{Quantoren}
                \label{ss:grundbegriffe_aussagen_quantoren}

                \begin{itemize}
                    \item $ \forall x \in M : E(x) $ - Für alle $ x $ in $ M $ gilt $ E(x) $ wobei $ E $ eine Aussageform darstellt.
                    \item $ \exists x \in M : E(x) $ - Es existiert mindestens ein $ x $ in $ M $ für das gilt $ E(x) $ wobei $ E $ eine Aussageform darstellt.
                \end{itemize}

                Beispiele:
                \begin{itemize}
                    \item $ \forall x \in \mathbb{R} : x ^ 2 >= 0 $ - (w)
                    \item $ \forall n \in \mathbb{N} : E _ 3 (n) $ - (f)
                    \item $ \exists n \in \mathbb{N} : E _ 3 (n) $ - (w)
                \end{itemize}

            \subsection{Aussagenlogische Verknüpfungen}
                \label{ss:grundbegriffe_aussagen_verknuepfungen}

                \begin{itemize}
                    \item $ A \land B $ - Konjunktion (und)
                    \item $ A \lor B $ - Disjunktion (oder)
                    \item $ A \implies B $ - Implikation (aus $ A $ folgt $ B $)
                    \item $ \lnot A $ - Negation (nicht)
                    \item $ A \iff B $ - Äquivalenz (Gleichheit)
                \end{itemize}

                \begin{tabular}{ c | c | c | c | c | c | c | c}
$ A $   & $ B $ & $ \lnot A $   & $ \lnot B $   & $ A \land B $ & $ A \lor B $  & $ A \implies B $ ($ (\lnot A) \lor B $)   & $ A \iff B $  \\
\hline
w       & w     & f             & f             & w             & w             & w                                         & w             \\
w       & f     & f             & w             & f             & w             & f                                         & f             \\
f       & w     & w             & f             & f             & w             & w                                         & f             \\
f       & f     & w             & w             & f             & f             & w                                         & f             \\
                \end{tabular}

                \begin{itemize}
                    \item[Äquivalenz] $ A \iff B \equiv (A \implies B) \land (B \implies A) $
                    \item[Kontraposition] $ A \implies B \iff (\lnot B \implies \lnot A) $
                \end{itemize}


                \subsubsection{de Morgan'schen Regeln}
                    \label{sss:grundbegriffe_aussagen_verknuepfungen_deMorgan}

                    \begin{itemize}
                        \item $ \lnot (A \lor B) \iff \lnot A \land \lnot B $
                        \item $ \lnot (A \land B) \iff \lnot A \lor \lnot B $
                    \end{itemize}


                \subsubsection{Distributivgesetz}
                    \label{sss:grundbegriffe_aussagen_verknuepfungen_distributivgesetz}

                    \begin{itemize}
                        \item $ (A \lor B) \land C \iff (A \land C) \lor (B \land C) $
                        \item $ (A \land B) \lor C \iff (A \lor C) \land (B \lor C) $
                    \end{itemize}



        \section{Mengen}
            \label{s:grundbegriffe_mengen}

            Beispiele:
            \begin{itemize}
                \item $ \mathbb{N} = \{ 0; 1; ...; n; ... \} $
                \item $ \mathbb{N*} = \{ 1; 2; ...; n; ... \} = \{ n \in \mathbb{N} : n \neq 0 \} $
                \item $ \{ x \in M : E(x) \} $ wobei $ E $ eine Aussagenform darstellt.
                \item $ \{ n \in \mathbb{N} : prim(x) \land n <= 6 \} = \{ 2; 3; 5 \} $
            \end{itemize}

            \subsection{Formalia}
                \label{ss:grundbegriffe_mengen_formalia}

                \begin{itemize}
                    \item $ A \subseteq B \equiv \forall x \in A : x \in B $
                    \item $ A = B \equiv (A \subseteq B) \land (B \subseteq A) \equiv \forall x \in M : (x \in A \implies x \in B ) \land (x \in B \implies x \in A) $
                    \item $ \emptyset \equiv \{ x \in A : x \neq x \} $ ($ x \neq x \equiv \lnot x = x $)
                \end{itemize}


            \subsection{de'Morganschen Regeln}
                \label{ss:grundbegriffe_mengen_demorgan}

                \begin{itemize}
                    \item $ (A \cup B) ^ \mathsf{c} = A ^ \mathsf{c} \cap B ^ \mathsf{c} $
                \end{itemize}


            \subsection{Kardinalität}
                \label{ss:grundbegriffe_mengen_kradinalitaet}

                Seien $ A $ und $ B $ endliche Mengen.

                Anzahl der Elemente (Kardinalität): $ \vert A \vert $

                \begin{itemize}
                    \item $ \vert A \cup B \vert = \vert A \vert + \vert B \vert - \vert A \cap B \vert $
                    \item $ \vert A \times B \vert = \vert A \vert * \vert B \vert $
                \end{itemize}

                $ \vert A \cup B \vert = \vert A \vert + \vert B \vert $ wenn $ A \cap B = \emptyset $


            \subsection{Operationen}
                \label{ss:grundbegriffe_mengen_operationen}

                $ M, N \in G $

                \begin{itemize}
                    \item $ M \cap N \equiv \{ x \in M : x \in N \} \equiv \{ x \in G : x \in M \land x \in N \} $
                    \item $ M \cup N \equiv \{ x \in G : x \in M \lor x \in N \} $
                    \item $ M \setminus N \equiv \{ x \in M : x \not\in N \} \equiv \{ x \in M : \lnot x \in N \} $
                    \item $ M ^ \mathsf{c} \equiv \{ x \in G : x \not\in M \} \equiv \{ x \in G : \lnot x \in M \} $
                    \item $ M \times N \equiv \{ (x, y) : x \in M , y \in N \} $ - Kartesisches Produkt
                    \item $ A _ 1 \times ... \times A _ n \equiv \{ (x _ 1, ..., x _ n) : x _ y \in A _ 1, ..., x _ n \in A _ n \} $
                    \item $ P(M) = \{ x : x \in M \} $
                    \begin{itemize}
                        \item $ \emptyset \subseteq P(\emptyset) \subseteq P(P(\emptyset)) \subseteq ... $
                        \item $ V _ w \subseteq P ^ n (\emptyset) $ ($ n \in \mathbb{N} $)
                        \item $ P(V _ w) = V _ (w + 1) $
                    \end{itemize}
                \end{itemize}


            \subsection{Obere/Untere Schranken}
                \label{ss:grundbegriffe_mengen_schranken}

                \begin{itemize}
                    \item[Obere Schranken:] $ OS(Y) = \{ x \in X : \forall y \in Y : x \geq y \} $
                    \item[Untere Schranken:] $ US(Y) = \{ x \in X : \forall y \in Y : x \leq y \} $
                \end{itemize}

                \begin{itemize}
                    \item[Supremum:] Das kleinste Element von $ OS(Y) $ $ \iff sup(Y) $.
                    \item[Infimum:] Das größte Element von $ US(X) $ $ \iff inf(Y) $.
                \end{itemize}


                \subsubsection{Beispiel}
                    $ \mathbb{Q} ^ \mathsf{+} = \{ x \in \mathbb{Q} : 0 < x \} $

                    \begin{itemize}
                        \item[Supremum:] Nicht vorhanden.
                        \item[Infimum:] $ US(\mathbb{Q} ^ \mathsf{+}) = \{ x \in \mathbb{Q} : x \leq 0 \} \implies inf(\mathbb{Q} ^ \mathsf{+}) = 0 $
                    \end{itemize}


            \subsection{Relationen}
                \label{ss:grundbegriffe_mengen_relationen}

                $ R \subseteq A _ 1 \times ... \times A _ n $

                \subsubsection{Relationen von identischen Mengen}

                    $ A ^ n = A \times ... \times A $ ($ n $ mal)

                    Für $ n = 2 $ kann die Infixnotation verwendet werden, das heißt $ xRy \iff (x, y) \in R $.


                \subsubsection{Definition von kleiner-gleich}

                    $ \leq = \{ (n, m) \if \mathbb{N} ^ 2 : n \leq m \} $


                \subsubsection{Eigenschaften}
                    \label{sss:grundbegriffe_mengen_relationen_eigenschaften}

                    \begin{itemize}
                        \item[Reflexivität] $ \forall x \in M : xRx $
                        \item[Symmetrie] $ xRy \implies yRx $
                        \item[Transivität] $ xRy \land yRz \implies xRz $
                        \item[Antisymmetrie] $ xRy \land yRx \implies x = y $
                    \end{itemize}

                    \begin{itemize}
                        \item R ist eine Äquivalenzrelation $ \iff $ R reflexiv, transitiv und symmetrisch
                        \item R ist eine partitielle Ordnung $ \iff $ R reflexiv, transitiv, antisymmetrisch
                        \item R ist total $ \iff $ $ \forall x y \in M : xRy \lor yRx $
                    \end{itemize}


            \subsection{Ordnungsrelationen}
                \label{ss:grundbegriffe_mengen_ordnungensrelationen}

                \subsubsection{Ordnungstypen}
                    \label{sss:grundbegriffe_mengen_ordnungensrelationen_typen}

                    p.O. $ \coloneqq $ partielle Ordnung

                    \begin{itemize}
                        \item \textbf{Totale Ordnung:} Jedes Element ist mit jedem anderen vergleichbar.
                        \item \textbf{Partielle Ordnung:} Nicht jedes Element ist nicht mit jedem anderen vergleichbar.
                    \end{itemize}


                \subsubsection{Ordnungsäquivalenz}
                    \label{sss:grundbegriffe_mengen_ordnungensrelationen_aequivalenz}

                    $ (x, R) $ p.O. $ y \subseteq x \implies (y, R \cap (y \times x)) $ p.O.

                    \begin{itemize}
                        \item $ x \geq y \iff y \leq x $
                        \item $ x > y \iff x \geq y \land x \neq y \iff x \geq y \land \lnot (x = y) $
                        \item $ x < y \iff y > x $
                    \end{itemize}


                \subsubsection{Extreme}
                    \label{sss:grundbegriffe_mengen_ordnungensrelationen_extreme}

                    $ (x, \leq) $ p.O. $ y \subseteq x $

                    \begin{itemize}
                        \item $ g \in X $ größtes Element von $ X $ $ \iff \forall x \in X : x \leq g $
                        \item $ k \in X $ kleinstes Element von $ X $ $ \iff \forall x \in X : x \geq k $
                    \end{itemize}

                    Größe Elemente sind immer eindeutig.

                    \paragraph{Beweis}
                        Die größten Elemente sind immer eindeutig.

                        Seien $ g $ und $ g' $ die größten Elemente.
                        $ \implies g \leq g' \land g' \leq g \implies g = g $

                        \qed


            \subsection{Große Vereinigung/Schittmenge / Leere Menge}
                \label{ss:grundbegriffe_mengen_grosseVerSchn}

                \subsubsection{Allgemein}

                    Allgemein gilt für Teilmengen von Potenzmengen $ Y \subseteq P(M) $:
                    \begin{itemize}
                        \item $ sup(Y) = \bigcup Y = \bigcup\limits _ { A \in Y } A $
                        \item $ inf(Y) = \bigcap Y = \bigcap\limits _ { A \in Y } A $
                    \end{itemize}


                \subsubsection{Sonderfall}

                    Für die leere Teilmenge der Potenzmenge $ Y = \emptyset $, $ Y \subseteq P(M) $ gilt:
                    \begin{itemize}
                        \item $ OS(\emptyset) = US(\emptyset) = P(M) $
                        \item $ sup(Y) = \bigcup \emptyset = \emptyset $
                        \item $ inf(Y) = \bigcap \emptyset = M $
                    \end{itemize}


            \subsection{Äquivalenzrelation}
                \label{ss:grundbegriffe_mengen_aequivalenzrelationen}

                Es gilt $ a, b, c, k, l, n \in \mathbb{Z} $.

                $ a \sim _ n b $ genau dann wenn $ \exists k \in \mathbb{Z} : a - b = k * n $

                \paragraph{Beweis}
                    Symmetrie

                    $ a - b = k * n \implies b - a = (-k) * n $

                    \qed


                \paragraph{Beweis}
                    Transitivität

                    $ a - b = k * n $, $ b - c = l * n \implies a - c = (a - b) + (b - c) = k * n + l * n = (k + l) * n $

                    \qed


            \subsection{Äquivalenzklassen}
                \label{ss:grundbegriffe_mengen_aequivalenzklassen}

                Es gilt $ (X, R) $, $ a \in X $.

                \begin{itemize}
                    \item $ a \in X $
                    \item $ \tilde{a} \coloneqq \{ x \in X : a \sim x \} $
                    \item $ \tilde{a} \neq \emptyset $
                    \item $ \bigcup \tilde{a} = X $
                    \item $ \tilde{a} \neq \tilde{b} \implies \tilde{a} \cap \tilde{b} = \emptyset $
                \end{itemize}

                \paragraph{Beweis}
                    $ \tilde{a} \neq \tilde{b} \implies \tilde{a} \cap \tilde{b} = \emptyset \equiv \tilde{a} \cap \tilde{b} \neq \emptyset \implies \tilde{a} = \tilde{b} $

                    Sei $ c \in \tilde{a} \cap \tilde{b} $, das heißt $ cRa $ und $ cRb $, also $ a \sim b $ und somit $ \tilde{a} = \tilde{b} $ und somit $ \tilde{a} \neq \tilde{b} \implies \tilde{a} \cap \tilde{b} = \emptyset $.

                    \qed


            \subsection{Partitionen}
                \label{ss:grundbegriffe_mengen_partitionen}

                $ P \subseteq P(X) $ ist genau dann eine Partition, wenn:
                \begin{itemize}
                    \item $ \bigcup P = X $
                    \item $ \forall A \in P : A \neq \emptyset $
                    \item $ \forall S _ 1 S _ 2 \in P : S _ 1 \neq S _ 2 \implies S _ 1 \cap S _ 2 = \emptyset $
                \end{itemize}

                \textbf{Äquivalenz:} $ x \sim _ p y \iff \exists S \in P : x \in A \land y \in A $

                $ X _ { / _ { P } } = P $

                $ x \sim _ { X _ { / _ { \sim } } } y \iff x \sim y $

                $ \frac{a}{b} \sim \frac{c}{d} \iff a * b \sim c * d $


        \section{Abbildungen/Funktionen}
            \label{s:grundbegriffe_funktionen}

            $ f : A \rightarrow B $ gdw. $ f \subseteq A \times B $, so dass
            \begin{itemize}
                \item $ xfy \land xfy' \implies y = y' $
                \item $ \forall x \in A : \exists y \in B : xfy $
            \end{itemize}

            $ f = graph(f) = \{ (x, f(x)) : x \in A \} $

            \begin{itemize}
                \item[$ C \subseteq A $] : $ f(C) = f[C] = \{ f(x) : x \in C \} $ (Bild von $ C $ unter $ f $).
                \item[$ D \subseteq B $] : $ f ^ { -1 } (D) = f ^ { -1 } [D] = \{ x \in A : f(x) \in D \} $
            \end{itemize}

            \subsection{Umkehrfunktion (inverse Funktion)}
                \label{ss:grundbegriffe_funktionen_umkehr}

                \textbf{Vorraussetzung zur Bildung einer inversen Funktion:} Die Funktion muss bijektiv sein.

                Sei $ f : A \rightarrow B $ bijektiv.

                Somit gilt für die Umkehrfunktion $ f ^ { -1 } = \{ (f(x), x) : x \in A \} : B \rightarrow A $

                Beziehungsweise $ R \in A \times B $, $ R ^ { -1 } = \{ (y, x) \in B \times A : xRy \} $


            \subsection{Identitätsfunktion}
                \label{ss:grundbegriffe_funktionen_identitaet}

                Sei $ M $ eine Menge.

                Für die Identitätsfunktion gilt: $ id _ M : M \rightarrow M : x \mapsto x $


            \subsection{Notation}
                \label{ss:grundbegriffe_funktionen_notation}

                Im allgemeinen gilt $ f : A \rightarrow B : x \mapsto f(x) $. Wobei $ A $ den Definitionsbereich und $ B $ den Wertebereich darstellt.

                Beispiele:
                \begin{itemize}
                    \item $ f : x \rightarrow x ^ 2 $
                    \item $ add : \mathbb{R} \times \mathbb{R} : (x, y) \mapsto x + y $
                    \item $ id _ A : A \rightarrow A : x \mapsto x $
                    \item Sei $ A $ eine Menge und $ \sim $ eine Äquivalenzrelation auf dieser. $ \mu : A \rightarrow A _ { /\sim } : x \mapsto \tilde{x} $
                    \item $ f : \mathbb{R} \rightarrow \mathbb{R} : x \mapsto x ^ 2 $
                    \item $ f : \mathbb{R} \rightarrow [0, \infty) : x \mapsto x ^ 2 $
                    \item $ f : [0, \infty) \rightarrow [0, \infty) : x \mapsto x ^ 2 $ (bijektiv) $ f ^ { -1 } : [0, \infty) \rightarrow [0, \infty) : y \mapsto \sqrt{y} $
                    \item $ f : [1, \infty) \rightarrow (0, 1] : x \mapsto \frac{1}{x} $ (bijektiv) $ f ^ { -1 } : (0, 1] \rightarrow [1, \infty) : x \mapsto \frac{1}{x} $
                \end{itemize}


            \subsection{Eigenschaften}
                \label{ss:grundbegriffe_funktionen_eigenschaften}

                \begin{itemize}
                    \item $ f $ ist injektiv, wenn $ \forall x, x' \in A : f(x) = f(x') \implies x = x' $.
                    \item $ f $ ist surjektiv, wenn $ \forall y \in B : \exists x \in A : f(x) = y $.
                    \item $ f $ ist bijektiv, wenn $ \forall y \in B : \exists ^ 1 x \in A : f(x) = y $.
                \end{itemize}

                Für jede Funktion $ f : A \rightarrow B $ existiert eine Funktion $ f ^ \# : A \rightarrow f[A] : x \mapsto f(x) $.


            \subsection{Funktionskomposition}
                \label{ss:grundbegriffe_funktionen_verkettung}

                Sei $ f : A \rightarrow B $ und $ g : B \rightarrow C $

                Durch die Verkettung entsteht eine neue Funktion: $ g \circ f : A \rightarrow C : x \mapsto g(f(x)) $

                Außerdem gilt:
                \begin{itemize}
                    \item $ f ^ { -1 } \circ f = id _ A $
                    \item $ f \circ f ^ { -1 } = id _ B $
                \end{itemize}


        \section{Beweisprinzipien}
            \label{s:grundbegriffe_beweise}

            \subsection{Direkter Beweis}
                \label{ss:grundbegriffe_beweise_direkt}

                Bei einem dirkten Beweis wird die Prämisse direkt bewiesen.

                \subsubsection{Beispiel}

                    Sind $ n, m \in \mathbb{N} $ gerade, dann ist $ n + m $ gerade.

                    \paragraph{Beweis:}

                        Es gilt $ n = 2 * k $ und $ m = 2 * l $, wobei $ k, l \in \mathbb{N} $. Das heißt, dass \[ n + m = 2 * k + 2 * l = 2 * (k + l) \] gerade ist.

                        \qed


            \subsection{Beweis durch Kontraposition}
                \label{ss:grundbegriffe_beweise_kontraposition}

                Anstelle von $ A \implies B $ wird $ \lnot B \implies \lnot A $ bewiesen.

                \subsubsection{Beispiel}

                    Gilt für $ n \in \mathbb{N} $, dass $ n ^ 2 $ gerade ist, ist $ n $ gerade.

                    \paragraph{Beweis:}

                        Der Beweis wird über $ n $ ungerade $ \implies n ^ 2 $ ungerade geführt.

                        Es gilt für $ k \in \mathbb{N} $, dass $ n = 2 * k + 1 $. Somit gilt dass \[ n ^ 2 = (2 * k + 1) ^ 2 = 4 * k ^ 2 + 4 * k + 1 = 2 * (2 * k ^ 2 + 2 * k) + 1 \] ungerade ist.

                        Daraus folgt dass $ n $ ungerade $ \iff n ^ 2 $ ungerade und $ n $ gerade $ \iff n ^ 2 $ gerade.

                        \qed


            \subsection{Indirekter Beweis}
                \label{ss:grundbegriffe_beweise_indirekt}

                Anstelle von $ A \implies B $ wird $ \lnot (A \land \lnot B) $ bewiesen. Alternativ kann $ \lnot (\lnot A) $ anstelle von $ A $ bewiesen werden ($ \lnot A \implies \bot $).

                \subsubsection{Beispiel}

                    $ \sqrt{2} $ is irrational.

                    \paragraph{Beweis:}

                        Ist $ \sqrt{2} $ rational, muss \[ \sqrt{2} = \frac{n}{m} \] ($ n, m \in \mathbb{N} $) gelten wobei $ n $ und $ m $ teilerfremd sind. Somit gilt \[ 2 = \frac{n ^ 2}{m ^ 2} \] also \[ n ^ 2 = 2 * m ^ 2 \]. Somit gilt $ n ^ 2 $ gerade $ \implies $ $ n $ gerade. Daraus folgt dass \[ (2 * k) ^ 2 = n ^ 2 = 2 * m ^ 2 \]. Somit gilt $ n ^ 2 $ gerade $ \implies $ $ n $ gerade.

                        $ \lightning $ $ n $ und $ m $ sollten teilerfremd sein. Somit ist $ \sqrt{2} \neq \frac{n}{m} $.

                        \qed


            \subsection{Beweis durch vollständige Induktion}
                \label{ss:grundbegriffe_beweise_induktion}

                Es wird beweisen, dass für eine \textit{Induktionshypothese} (\textit{IH}) $ A(n) $ gilt \[ (A(0) \land (\forall n \in \mathbb{N} : A(n) \implies A(n + 1))) \implies \forall n \in \mathbb{N} : A(n) \]
                Der Beweis von $ A(0) $ wird \textit{Induktionsanfang} (\textit{IA}) genannt.

                Der Beweis von $ A(n) \implies A(n + 1) $ wird \textit{Induktionsschritt} (\textit{IS}) genannt.

                Es gilt: \[ A(0) \implies A(1) \implies A(2) \implies \dots \implies A(n) \]

                \subsubsection{Beispiel 1}

                    \[ \forall n \in \mathbb{N} : \sum _ { k = 1 } ^ n k = \frac{n * (n + 1)}{2} \]

                    \paragraph{Beweis:}

                        Im folgenden wird die Gaußsche Summenformel mit Hilfe der vollständigen Induktion bewiesen.

                        \textbf{Induktionsanfang:} \[ A(0) = \sum _ { k = 1 } ^ 0 k = \frac{0 * (0 + 1)}{2} = 0 \]

                        \textbf{Induktionsschritt:} 
                        \begin{eqnarray*}
                            A(n + 1) = \sum _ { k = 1 } ^ { n + 1 } = (\sum _ { k = 1 } ^ n) + (n + 1)
                                && = \frac{n * (n + 1)}{2} + (n + 1)                                    \\
                                && = \frac{n + (n + 1)}{2} + \frac{2 * (n + 1)}{2}                      \\
                                && = \frac{n * (n + 1) + 2 * (n + 1)}{2}                                \\
                                && = \frac{n ^ 2 + 3 * n + 2}{2}                                        \\
                                && = \frac{(n + 1) * (n + 2)}{2}                                        \\
                        \end{eqnarray*}

                        \qed


                \subsubsection{Beispiel 2}

                    Für endliche Mengen $ M $ gilt $ \vert P(M) \vert = 2 ^ { \vert P(M) \vert } $, also \[ \forall n \in \mathbb{N} : \forall M ; \vert M \vert = n \implies \vert P(M) \vert = 2 ^ n \].

                    \paragraph{Beweis:}

                        Im folgenden wird oben genannte Prämisse mit Hilfe der vollständigen Induktion bewiesen.

                        \textbf{Induktionsanfang:} \[ \vert M \vert = 0 \implies \vert P(M) \vert = 2 ^ 0 = 1 \]

                        \textbf{Induktionsschritt:} Sei $ M $ eine Menge mit $ \vert M \vert = n + 1 $ wobei $ n \in \mathbb{N} $. Zu zeigen: $ \vert P(M) \vert = 2 ^ { n + 1 } = 2 * 2 ^ n $.

                        Sei $ a \in M $.

                        \begin{itemize}
                            \item $ S _ 0 = \{ A \in P(M) : a \in A \} \implies \vert S _ 0 \vert = n $
                            \item $ S _ 1 = \{ A \in P(M) : a \not\in A \} \implies \vert S _ 1 \vert = n $
                        \end{itemize}

                        \[ S _ 0 \approx P(M \setminus \{ a \}) \approx S _1 \]

                        \[ \vert P(M) \vert = \vert S _ 0 \vert + \vert S _ 1 \vert = 2 * \vert P(M \setminus \{ a \}) \vert = 2 * 2 ^ n = 2 ^ { n + 1 } \]

                        \qed



    \chapter{Algebraische Strukturen}
        \label{c:strukturen}

        Strukturen, in denen man "`wie üblich"' rechnen kann.

        \section{Rechnen in $ \mathbb{Z} $ - Primzahlen, Teiler}
            \label{s:strukturen_rechnen}

            Seien $ a, b \in \mathbb{Z} $.

            \begin{itemize}
                \item $ b \mid a \equiv \exists c \in \mathbb{Z} : b * c = a \equiv $ $ b $ teilt $ a $.
                \item $ p \in \mathbb{N} $ ist prim $ \iff (p > 1) \land (\forall n \in \mathbb{N} : n \mid p \implies (n = 1 \lor n = p)) $
                \item $ ggt(a, b) = max(\{ n \in \mathbb{N} : n \mid a \land n \mid b \}) $
            \end{itemize}

            Sei $ a \in \mathbb{Z} $ und $ b \in \mathbb[N*] $, dann existieren eindeutige $ q \in \mathbb{Z} $ und $ r \in \{ 0, 1, \dots, b - 1 \} $ mit $ a = q * b + r $ wobei $ q = \lfloor \frac{a}{b} \rfloor $ und $ r = a \mod b $.
\end{document}
