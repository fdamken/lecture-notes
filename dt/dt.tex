%%
% Copyright 2016 Fabian Damken
% 
% Licensed under the Apache License, Version 2.0 (the "License");
% you may not use this file except in compliance with the License.
% You may obtain a copy of the License at
% 
%     http://www.apache.org/licenses/LICENSE-2.0
% 
% Unless required by applicable law or agreed to in writing, software
% distributed under the License is distributed on an "AS IS" BASIS,
% WITHOUT WARRANTIES OR CONDITIONS OF ANY KIND, either express or implied.
% See the License for the specific language governing permissions and
% limitations under the License.
%%



\documentclass[12pt]{scrreprt}

\usepackage{amsfonts}
\usepackage[german]{babel}
\usepackage{eqnarray}
\usepackage[T1]{fontenc}
\usepackage[utf8]{inputenc}
\usepackage{listingsutf8}
\usepackage{mathtools}
\usepackage{stmaryrd}
\usepackage{tikz}
\usetikzlibrary{arrows}

\title{Digitaltechnik}
\subtitle{Mitschrift}
\author{Fabian Damken}
\date{\today}

\setcounter{secnumdepth}{3}

\newcommand{\qed}{\begin{flushright}q.e.d.\end{flushright}}

\begin{document}
    \maketitle
    \tableofcontents

    \chapter{Beherrschen der Komplexität}
        \label{c:komplexitaet}

        \begin{itemize}
            \item Abstraktion
            \item Disziplin
            \item
                Die der Ypsilons
                \begin{itemize}
                    \item Hierarchy (Hierarchie)
                    \item Modularity (Modularität)
                    \item Regularity (Regularität)
                \end{itemize}
        \end{itemize}

        \begin{tabular}{c | c | c | c}
                & Schicht               & Teile                 & Fachbereich           \\
            \hline
            1   & Anwendungssoftware    & Programme             & Software Engineering  \\
            2   & Betriebssystem        & Geräretreiber         & Software Engineering  \\
            3   & Architektur           & Befehle, Register     & Rechnerorganisation   \\
            4   & Mikroarchitektur      & Datenpfade, Steuerung & Rechnerorganisation   \\
            5   & Logik                 & Addierer, Speicher    & Digitaltechnik        \\
            6   & Digitalschaltungen    & Gatter, Inverter      & Digitaltechnik        \\
            7   & Analogschaltungen     & Verstärker, Filter    & Elektrotechnik        \\
            8   & Bauteile              & Transistoren, Dioden  & Elektrotechnik        \\
            9   & Physik                & Elektronen            & Physik                \\
        \end{tabular}
\end{document}
