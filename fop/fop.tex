%%
% Copyright 2016 Fabian Damken
% 
% Licensed under the Apache License, Version 2.0 (the "License");
% you may not use this file except in compliance with the License.
% You may obtain a copy of the License at
% 
%     http://www.apache.org/licenses/LICENSE-2.0
% 
% Unless required by applicable law or agreed to in writing, software
% distributed under the License is distributed on an "AS IS" BASIS,
% WITHOUT WARRANTIES OR CONDITIONS OF ANY KIND, either express or implied.
% See the License for the specific language governing permissions and
% limitations under the License.
%%



\documentclass[12pt]{scrreprt}

\usepackage{amsfonts}
\usepackage[german]{babel}
\usepackage[T1]{fontenc}
\usepackage[utf8]{inputenc}
\usepackage{mathtools}

\title{Funktionale und objektorientierte Programmierung}
\subtitle{Mitschrift}
\author{Fabian Damken}
\date{\today}

\setcounter{secnumdepth}{3}

\newcommand{\qed}{\begin{flushright}q.e.d.\end{flushright}}

\begin{document}
    \maketitle
    \tableofcontents

    \chapter{Programmierung}
        \label{c:programmierung}

        \section{Strukturmechanismen}
            \label{s:programmierung_strukt}

            \begin{itemize}
                \item
                    Primitive Ausdrücke
                    \begin{itemize}
                        \item Zahlen (selbstauswertend)
                        \item Boolsche Werte (selbstauswertend)
                        \item Eingebaute Prozeduren (bspw. \texttt{+}, \texttt{-}, \texttt{*}, \texttt{/}, ...)
                    \end{itemize}
                \item Kombinationsmittel
                \item Abstraktionsmittel
            \end{itemize}


        \section{Programmentwicklung}
            \label{s:programmierung_entwicklung}

            \begin{enumerate}
                \item Zweck verstehen.
                \item Beispiel ausdenken.
                \item Implementieren.
                \item Testen.
            \end{enumerate}


    \chapter{DrRacket}
        \label{c:drracket}

        \section{Strukturmechanismen}
            \label{s:drracket_struktur}

            \subsection{Primitive Sprachelemente}
                \label{ss:drracket_struktur_primitive}

                \begin{itemize}
                    \item Ganze Zahlen [bspw. \texttt{3}]
                    \item Endliche Dezimalzahlen (Punkt statt Komma) [bspw. \texttt{2.3}]
                    \item Brücke (ohne Leerzeichen) [bspw. \texttt{7/3}]
                    \item Unendliche Zahlen ohne Periode (beginnend mit \texttt{\#i}) [bspw. \texttt{\#i2.718}]
                \end{itemize}


            \subsection{Kombinationsmittel}
                \label{ss:drracket_struktur_kombination}

                Kombinationen werden in DrRacket mit Hilfe der Präfixdarstellung dargestellt. Das heißt dass der gesamte Ausdruck in Klammern gefasst wird, erst der Operator und dann mögliche Operanden getrennt mit einem Leerzeichen.

                Beispiele:
                \begin{itemize}
                    \item \texttt{(+ 2 3)}
                    \item \texttt{(+ 4 (* 2 3))}
                \end{itemize}

                Kombination $ \iff $ Prozedur


            \subsection{Abstraktionsmittel}
                \label{ss:drracket_struktur_abstraktion}

                \begin{enumerate}
                    \item Erstellen einer komplexen Sache.
                    \item Benennung der komplexen Sache.
                    \item Nutzung als primitive Sache.
                \end{enumerate}

                Beispiele:
                \begin{itemize}
                    \item \texttt{(define score (+ 27 3))}
                    \item \texttt{(define PI 3.1415)}
                \end{itemize}


            \subsection{Auswertungsregeln}
                \label{ss:drracket_struktur_auswertung}

                \begin{itemize}
                    \item Selbstauswertend (self-rule) [bspw. \texttt{2}, \texttt{true}]
                    \item Eingebauter Operator [bspw. \texttt{+}, \texttt{*}]
                    \item Name/Variablen (name-rule)
                    \item
                        Kombination [bspw. \texttt{(+ 2 3)}]
                        \begin{enumerate}
                            \item Werte Unserausdrücke aus.
                            \item Wende die Prozedur an. [Beispiel: \texttt{(+ 4 (* 2 3)) = (+ 4 6) = 10}]
                        \end{enumerate}
                    \item Sonderformen
                \end{itemize}


            \subsection{define - Regel}
                \label{ss:drracket_struktur_define}

                \begin{itemize}
                    \item Operand 1 $ \coloneqq $ Name
                    \item Operand 2 $ \coloneqq $ Wert
                    \item Wertet nur den Wert aus.
                    \item Der Wert wird an den Namen gebunden.
                    \item Kein Rückgabewert.
                \end{itemize}


            \subsection{Prozeduren}
                \label{ss:drracket_struktur_prozeduren}

                \subsubsection{Kreisfläche}
                    \label{sss:drracket_struktur_prozeduren_kreisflaeche}

                    Berechnet die Fläche eines Rings.

                    \textbf{Definition:} \texttt{(define (area-of-disk r) (* 3.14 (* r r)))}

                    \textbf{Nutzung:} \texttt{(area-of-disk 5) = 78.5}


                \subsubsection{Ringfläche}
                    \label{sss:drracket_struktur_prozeduren_ringflaeche}

                    Berechnet die Fläche eines Rings durch die Subtraktion der Fläche des Innenrings von der Fläche des Außenrings.

                    \textbf{Definition:} \texttt{(define (area-of-ring outer inner) (- (area-of-disk outer) (area-of-disk inner)))}

                    \textbf{Nutzung:} \texttt{(area-of-ring 5 3) = 50.24}


        \section{Programmentwicklung}
            \label{s:drracket_entwicklung}

            \subsection{Zweck verstehen}
                \label{ss:drracket_entwicklung_zweck}

                \begin{lstlisting}
;; Funktionsname :: Parametertypen -> Ergebnistyp
;; Kurze Beschreibung.
(define ...)
                \end{lstlisting}

            \subsection{Beispiel ausdenken}
                \label{ss:drracket_entwicklung_beispiel}

                \begin{lstlisting}
;; Beispiel: (Funktionsname Beispielparameter) = Beispielergebnis
                \end{lstlisting}

            \subsection{Testen}
                \label{ss:drracket_entwicklung_testen}

                \begin{itemize}
                    \item \texttt{(check-expect test expected)} - Prüfung auf exakte Gleichheit.
                    \item \texttt{(check-within test expected delta)} - Prüfung auf ungefähre Gleichheit in dem gegebenen Toleranzramen (delta).
                    \item \texttt{(check-error test message)} - Prüfung auf das Auslösen eines Fehlers.
                \end{itemize}
\end{document}
